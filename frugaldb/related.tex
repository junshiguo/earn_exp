\section{Related Work}\label{sec:RelatedWork}

Database-as-a-service~(DBaaS) is becoming increasingly popular as a cost-effective data management model for providing services in multi-tenant cloud environments, and there have been extensive research efforts devoted to improving services for multi-tenancy to support mixed workloads in cloud environments w/o performance guarantees.

A lot of previous work focused on virtual machine~(VM) configuration, provisioning and consolidation to avoid resource over-provisioning~\cite{DRA,IMVR}. Soror et al.~\cite{AVMC} proposed a method to automatically configure VMs for different database workloads consolidated on the same hardware, and Soundararajan et al.~\cite{DRA} presented a system for determining resource partitions for virtual machines. Shen et al.~\cite{CloudScale} presented CloudScale to online predict resource demand of VM-based applications and automate fine-grained elastic resource scaling for virtualized cloud computing infrastructures, resolving scaling conflicts using VM migration. Some prior work looked at dynamic provisioning of VMs for multi-tier web applications, such as Cecchet et al.~\cite{Dolly} presented Dolly which explore VM cloning technique to spawn database replicas to address the challenges of provisioning shared-nothing replicated databases in clouds. Nathuji et al.~\cite{Q-clouds} presented Q-Clouds as a QoS-aware control framework to dynamically tune resource allocations to mitigate performance interference between applications consolidated on the same server. These VM-based consolidation approaches achieve moderate consolidation and are appropriate only when hard isolation between databases is more important than cost or performance. Schiller et al.~\cite{NSMT} proposed tenant-aware data management to natively support multi-tenancy in database. Narasayya et al. ~\cite{SQLVM,CPUS} presented SQLVM to provide isolated performance for multi-tenant DBaaS by reserving resources specified by tenants within database server process, instead of packing resources in the form of VMs. Curino et al.~\cite{Workload-Aware} and Lang et al.~\cite{TSLOS} presented mechanisms for consolidating multiple databases into the same server, aiming to minimize the number of servers needed to satisfy performance goals for a large number of workloads. Their work is complementary to ours, as it could be used to optimize resource partitions or assignment of workloads which may be generated by our scheme.

Some prior work~\cite{WIDE} is devoted to achieving extremely high levels of multi-tenancy, where common tables are shared by tens of thousands of tenants with identical or similar databases schemas, and yet with nearly inactive database workloads. Aulbach et al.~\cite{schema-mapping} proposed a schema-mapping multi-tenant technique, called Chunk Folding, where tenants' logical tables are vertically partitioned into chunks and folded into different shared physical multi-tenant tables, which may be joined as needed for query processing. Hui et al.~\cite{M-Store} presented M-Store to consolidate tuples from different tenants into shared tables, where Bitmap Interpreted Tuple~(BIT) and Multi-Separated Index~(MSI) are proposed to improve scalability and efficiency of multi-tenant database system. This prior work aims at resolving the key challenge of improving databases�� scalability when extremely large numbers of tables and/or columns need to be handled, and none takes performance SLOs into consideration. So this work is closely related to ours, yet complementary to our work, as our approach may also run into similar internal database limitations.

Some solutions focus on workload queuing and scheduling~\cite{PAaaS}, and admission control~\cite{ActiveSLA} for multi-tenancy, when performance SLOs are taken into consideration. These approaches follows the resource-reservation fashion, which do not increase real processing capacity of database servers and may decline services for tenants when data serving systems become overwhelmed. Their work is complementary to ours, as we could employ these approaches to implement traffic control when our scheme is not able to satiate tenants' performance SLOs anyway. Our main target is to handle high-intensity workloads in time, so as to eliminate violations of performance QoS guarantees during workload-bursty periods. 