\section{Introduction}\label{sec:Introduction}

The Internet has enabled and popularized DBaaS~\cite{PDBaaS,RMTD,salesforce,Force,MTDB1,DBMaaS,DBaaS}, where service providers host database applications on their infrastructures, and deliver them as services to different end users ~(also called tenants), who would otherwise need to deploy these database applications by themselves. Tenants who have subscribed DBaaS services could query and update their data via web browsers or client programs, so the ownership and operation of database applications are outsourced to DBaaS service providers. DBaaS service providers typically consolidate multiple tenants into the same hardware/software platform so as to reduce operational cost via resource sharing.

When tenants necessitate QoS guarantees and require satisfaction of performance SLOs, service providers usually have to reserve resources for tenants according to their performance SLOs, so that a database server would always have enough processing capacity to handle all tenants' workloads consolidated on this server, under various workload conditions. However, service providers could only achieve moderate resource utilizations in such a resource-reservation tenant consolidation fashion. DBaaS tenants often quantify their resource requirement according to the peaks of their workloads, and they usually tend to over-estimate their performance SLOs to make sure that subscribed resources could completely cover the actual resource needs of their businesses, so the workload pressure generated by a tenant will not be persistently intense enough to catch up with this tenant's performance SLO. What's more, multi-tenancy tends to possess low overall tenant activeness, and there exist few chances that massive tenants would generate their workloads at their full speeds concurrently, so the DBaaS system would witness moderate resource utilizations most of the time.

Carlo Curino et al.~\cite{Workload-Aware} proposed a buffer pool gauging procedure, included in their tenant consolidation scheme named Kairos for their multi-tenant database platform~\cite{Relational}, to estimate the working set size of a database workload, so that they could properly accomplish consolidation for multiple database workloads with minimum servers. However, this buffer pool gauging procedure is not applicable for our targeting scenario. The difficulty is twofold: firstly, to precisely quantify a tenant's resource demands requires the tenant's workload to be stable, while it is impossible to accomplish precise quantifications if the tenant's workload changes dynamically over time; secondly, when far more tenants are consolidated into the same database, there is no reliable mechanism for guaranteeing that memory resources are allocated to processing workloads which actually require more resources. As a huge amount of tenants are consolidated into the same database, buffer resources allocated to tenants with high-intensity workloads may be spared for other massive numbers of tenants with low-intensity workloads, and precious memory resources are not persistently employed to buffer data for those tenants with high-intensity workloads but wasted on buffering data for tenants with low-intensity workloads. So our targeting scenario requires an effective mechanism which could dynamically quarantine tenants with high-intensity workloads from tenants with low-intensity workloads, which could guarantee that processing of high-intensity workloads would not affect by low-intensity workloads, and thus the DBaaS system could get the most out of resources allocated to process high-intensity workloads.

Targeting at application scenarios where thousands of small tenants with low overall tenant activeness and yet with various database schemas should be served with QoS guarantees, we propose a novel multi-tenant database system called FrugalDB to further improve resource utilization and reduce operational cost, where small tenants' database sizes are expected to be at tens of or a few hundred megabytes scales and their workloads usually exhibit obvious unstableness with huge variance. FrugalDB could consolidate more tenants with performance SLOs onto the same database server. While for medium or large tenants whose database sizes grow up to gigabytes with strict QoS requirements and stable workloads, we think it would be much more appropriate to adopt the shared-machine scheme, which encapsulates different tenants' database in different properly configured virtual machines~(VMs) running independent databases, and it is out of this paper's discussion scope.

We mainly make the following three contributions in this paper: ~1) propose a workload offloading mechanism to handle mixed workloads of tenants with performance SLOs; ~2) formulate and solve the workload offloading problem as an optimization problem; ~3) implement a prototype of the proposed workload offloading mechanism, and perform extensive experiments to evaluate the efficiency of the mechanism. In the remainder of this paper, we first introduce the system design and implementation details about FrugalDB in Sec.~\ref{sec:SDI}, and formally define the targeting problem and present the algorithms employed to solve the problem in Sec.~\ref{sec:PFA}. We provide experimental results to validate FrugalDB's efficiency in Sec.~\ref{sec:Experiments}, and present the related work of FrugalDB in Sec.~\ref{sec:RelatedWork}. Sec.~\ref{sec:Conclusion} concludes this paper. 